\documentclass{article}
    \usepackage[utf8]{inputenc} %caratteri speciali tastiera come "ç"
    \usepackage{lmodern}
    \usepackage[T1]{fontenc} %codifica font 
    \usepackage{calligra}
    % \usepackage{emerald}
    \usepackage[italian]{babel} %lingua del documento
    \usepackage{graphicx}
    \usepackage{listings}
    \usepackage{color}
    \definecolor{lightgray}{gray}{0.9}
    \usepackage{geometry}    
    \setlength{\parindent}{5mm}
    \linespread{1.5}
    \geometry{a4paper, left=1.5cm, right=1.5cm, top=2.5cm, bottom=2cm}
    \usepackage{titlesec}
    \titleformat{\section}
    [hang]
    {\centering\bfseries\em\fontsize{35}{1}\selectfont}
    {\thesection.}{0.5em}{\vspace{0.5ex}}[\vspace{0.5ex}]
    \titleformat{\subsection}
    [hang]
    {\bfseries\large\sffamily}
    {\thesection.}{0.5em}{}
    % \usefont{<encoding>}{<family>}{<series>}{<shape>}
    % \usefont{T1}{caligrafa}{}{} font corsivo per firme
    % \includegraphics[scale=0.6]{immagine} per inserire immagini
    % \fontsize{}{}\selectfont per selzionare la grandezza del font
    % \vfill per il fondo pagina nella pagina del titolo
    % \par\vspace{3cm} serve per un nuovo paragrafo e l'interlinea

\begin{document}
\begin{titlepage}
    \centering
    \includegraphics[]{img/roImg.jpeg}
    \par\vspace{4cm}
    {\fontsize{60}{1}\selectfont\usefont{T1}{emerald}{}{} Ricerca Operativa}
    \par\vspace{3cm}
    {\usefont{T1}{calligra}{}{} \fontsize{50}{1}\selectfont Mihai Eni}
    \vfill
    {\large{2018/2019}}
    \end{titlepage}

\tableofcontents
\newpage

\section{Introduzione}
    \subsection{Cosa significa ottimizzare}
        Cercare di ottenere il miglior risultato con il minimo sforzo, impiegare meno tempo possibile per compiere qualche incarico assegnatoci e sfruttare al meglio le risorse messe a disposizione. Avendo un numero di componenti limitato è abbastanza semplice trovare tutte le soluzioni ammissibili e decidere qualle sia la migliore. Certamente è meno semplice riuscire a certificare quale sia la soluzione migliore in assoluto, quantificare il valore delle risorse o valutare la stabilità della soluzione di fronte a variazioni, senza l'utilizzo di strumneti matematici.
        
    \subsection{Ricerca Operativa}
        La ricerca operatica ha come oggetto lo studio e la messa a punto di metodologie e strumenti quantitativi per la soluzione di problemi decisionali. È Nata durante in ambito militare durante la seconda guerra mondiale e ha trovato applicazione in svariati settori. Attualmente la ricerca operativa è diventato uno strumento fondamentale per supportare i processi decisionali. I problemi affrontati sono detti di \textit{ottimizzazione}\footnote{un \textbf{problema di ottimizzazione} è il problema di trovare la migliore soluzione fra tutte le soluzioni fattibili}.
    
    \subsection{Come affrontare un problema di ottimizzazione}
        Durante la vita quotidiana siamo abituati a trovarci di fronte a problemi di ottimizzazione ad esempio trovare il percorso minimo per andare a lavoro. Ovviamente per risolvere i problemi quotidiani non c'è bisogno degli strumenti quantitativi messi a disposizione dalla ricerca operativa, ma per problemi più complessi e di più vasto impatto questi strumenti non possono essere. 
        \paragraph{Processo decisionale}
            \begin{itemize}
                \item individuazione del problema decisionale;
                \item analisi della realtà e raccolta dei dati;
                \item costruzione del modello come astrazione del problema reale;
                \item determinazione di una o più soluzioni;
                \item analisi dei risultati ottenuti e loro interpretazione nel caso reale.
                \end{itemize}
            Le 5 fasi non sono necessariamente in sequenza, anzi spesso i risultati di una fase suggeriscono una modifica delle scelte fatte in precedenza.
    \subsection{Modello}
        Il modello è una astrazione che permette di descrivere in termini "matematici" le caratteristiche salienti del problema che si vuole studiare e risolvere. 
        \paragraph{Formulazione di un modello}
            Nella fomulazione di un modello i seguenti 3 punti sono di fondamentale importanza:
            \begin{enumerate}
                \item individuazione delle \textit{decisioni che interessano il problema}. In un modello matematico le decisioni vengono rapresentate da \textbf{variabili decisionali};
                \item determinazione dell'\textit{obiettivo} o degli obiettivi da ottimizzare. In generale è possibile esprimere l'obiettivo come una funzione delle varibaili decisionali detta \textbf{funzione obiettivo};
                \item definizioni delle \textit{soluzioni ammissibili}. Si vogliono in pratica esprimere le "relazioni" che definiscono il problema decisionale. In genere queste relazioni vengono chiamate \textbf{vincoli del problema}.
            \end{enumerate}
        \paragraph{Classi di modelli}
            I modelli vengono suddivisi nelle seguenti tre classi principali:
            \begin{description}
                \item [modelli di programmazione matematica]:
                \item [modelli di teoria dei giochi]:
                \item [modelli di simulazione]:
                \end{description}
\end{document}
